\documentclass[10pt,twocolumn,letterpaper]{article}

\usepackage{iccv}
\usepackage{times}
\usepackage{epsfig}
\usepackage{graphicx}
\usepackage{amsmath}
\usepackage{amssymb}

% Include other packages here, before hyperref.

% If you comment hyperref and then uncomment it, you should delete
% egpaper.aux before re-running latex.  (Or just hit 'q' on the first latex
% run, let it finish, and you should be clear).
\usepackage[pagebackref=true,breaklinks=true,letterpaper=true,colorlinks,bookmarks=false]{hyperref}

% \iccvfinalcopy % *** Uncomment this line for the final submission

\def\iccvPaperID{408} % *** Enter the ICCV Paper ID here
\def\httilde{\mbox{\tt\raisebox{-.5ex}{\symbol{126}}}}

% Pages are numbered in submission mode, and unnumbered in camera-ready
\ificcvfinal\pagestyle{empty}\fi
\begin{document}

%%%%%%%%% TITLE
\title{Forget the checkerboard: practical self-calibration using a planar scene}

\author{First Author\\
Institution1\\
Institution1 address\\
{\tt\small firstauthor@i1.org}
% For a paper whose authors are all at the same institution,
% omit the following lines up until the closing ``}''.
% Additional authors and addresses can be added with ``\and'',
% just like the second author.
% To save space, use either the email address or home page, not both
\and
Second Author\\
Institution2\\
First line of institution2 address\\
{\tt\small secondauthor@i2.org}
}

\maketitle
%\thispagestyle{empty}


%%%%%%%%% ABSTRACT
\begin{abstract}
We introduce a self-calibration method using a planar scene of unknown texture. Planar surfaces are everywhere but checkerboards are not, thus the method can be easily applied outside of the lab. We demonstrate that the accuracy is equivalent to a checkerboard-based calibration, so there is no need for printing checkerboards any more. Moreover, the use of a planar scene provides improved robustness and stronger constraints than a self-calibration with an arbitrary scene. We provide a closed-form initialization of the focal length with minimal and practical assumptions. The method recovers the intrinsic and extrinsic parameters of the camera and the metric structure of the planar scene. The method is implemented in a real-time calibration application for non-expert users that provides an easy and practical process to obtain high accuracy calibrations. 
\end{abstract}

%%%%%%%%% BODY TEXT	
\section{Introduction}

Calibrating a camera's intrinsics is a fundamental problem in computer vision. A calibrated camera is needed to perform a metric reconstruction of a scene, otherwise only a projective reconstruction is possible \cite{hartley2000}. Some of the most interesting applications of computer vision, like simultaneous localization and mapping, augmented reality, and 3D reconstruction, require a metric reconstruction of the scene. Cameras are most often calibrated offline using a calibration target. Using a planar target with a checkerboard pattern of known structure is a well established and popular method for camera calibration \cite{zhang2000,bouguetMCT}.	

{\small
\bibliographystyle{ieee}
\bibliography{egbib}
}

\end{document}
